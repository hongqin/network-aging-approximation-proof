\documentclass{article}
\usepackage{amsmath}
\usepackage{amsfonts}

\begin{document}

\section*{\textbf{Proof: Approximating $(t_0 + t)^{n-1}$ by an Exponential Function}}

In Hong Qin's network model of aging, the mortality rate $\mu_{net}(t)$ is given by:
\[
\mu_{net}(t) \approx cmn(p\lambda)^n \left( \frac{1 - p}{p \lambda} + t \right)^{n-1}
\]
where $t_0 = \frac{1 - p}{p \lambda}$.

Simplifying this, we have:
\[
\mu_{net}(t) \approx cmn(p\lambda)^n (t_0 + t)^{n-1}
\]

To understand how this term can be approximated by an exponential function, leading to $\exp(Gt)$ in the $\mu_{net}(t)$ formula, let's follow these steps:

\section*{\textbf{Step-by-Step Approximation}}

1. \textbf{Express $t_0$ and $G$ in Terms of Network Parameters:}

   The virtual age $t_0$ and rate of increase in mortality $G$ are given by:
   \[
   t_0 = \frac{1 - p}{p \lambda}
   \]
   \[
   G = \frac{(n - 1) p \lambda}{1 - p}
   \]

2. \textbf{Simplified Form of $\mu_{net}(t)$:}

   Using the definition of $t_0$, we rewrite $\mu_{net}(t)$ as:
   \[
   \mu_{net}(t) \approx cmn(p\lambda)^n \left( \frac{1 - p}{p \lambda} + t \right)^{n-1}
   \]

   For small $t$ (i.e., $t \ll t_0$), we can approximate this as:
   \[
   \mu_{net}(t) \approx cmn(p\lambda)^n \left( \frac{1 - p}{p \lambda} \right)^{n-1}
   \]

3. \textbf{Approximation by Exponential Function:}

   The term $(t_0 + t)^{n-1}$ can be approximated by an exponential function for large $t$ using the following steps:
   
   - Consider the approximation $(t_0 + t) \approx t_0 \cdot \left(1 + \frac{t}{t_0}\right)$.
   - Using the binomial expansion for large $t$:
     \[
     (t_0 + t)^{n-1} \approx t_0^{n-1} \left(1 + \frac{t}{t_0}\right)^{n-1}
     \]
   - For large $t$, $\left(1 + \frac{t}{t_0}\right)^{n-1}$ can be approximated using the exponential function:
     \[
     \left(1 + \frac{t}{t_0}\right)^{n-1} \approx \exp\left( (n-1) \ln\left(1 + \frac{t}{t_0}\right) \right)
     \]

   For large $t$:
   - $\ln\left(1 + \frac{t}{t_0}\right) \approx \frac{t}{t_0}$ because $\ln(1 + x) \approx x$ for small $x$.
   - Thus:
     \[
     \exp\left( (n-1) \ln\left(1 + \frac{t}{t_0}\right) \right) \approx \exp\left( (n-1) \frac{t}{t_0} \right)
     \]

   So we have:
   \[
   (t_0 + t)^{n-1} \approx t_0^{n-1} \exp\left( (n-1) \frac{t}{t_0} \right)
   \]

4. \textbf{Incorporate into $\mu_{net}(t)$:}

   Substitute this back into the mortality rate expression:
   \[
   \mu_{net}(t) \approx cmn(p\lambda)^n t_0^{n-1} \exp\left( (n-1) \frac{t}{t_0} \right)
   \]

   Recall $t_0 = \frac{1 - p}{p \lambda}$:
   \[
   \mu_{net}(t) \approx cmn(p\lambda)^n \left( \frac{1 - p}{p \lambda} \right)^{n-1} \exp\left( (n-1) \frac{t p \lambda}{1 - p} \right)
   \]

   Simplify the constants:
   \[
   \mu_{net}(t) \approx R \exp\left( G t \right)
   \]
   where:
   \[
   R = cmn(p\lambda)^n \left( \frac{1 - p}{p \lambda} \right)^{n-1}
   \]
   and:
   \[
   G = \frac{(n-1) p \lambda}{1 - p}
   \]

This shows how the original term $(t_0 + t)^{n-1}$ can be approximated by an exponential function, leading to the $\exp(Gt)$ term in the $\mu_{net}(t)$ formula.

\end{document}
